\capitulo{7}{Conclusiones y Líneas de trabajo futuras}

Para dar cierre a esta memoria se expondrán en esta sección las conclusiones obtenidas tras el desarrollo del proyecto y posibles futuras líneas de trabajo a realizar.

\section{Conclusiones}
De este trabajo se han sacado las siguientes conclusiones:

\begin{itemize}
\item Se ha logrado el objetivo de desarrollar una plataforma orientada a la enseñanza a través de cursos con tareas en Python corregibles de forma automática, junto con la posibilidad de diseñar estas tareas desde la propia plataforma. Además se ha incluido en esta un pequeño curso introductorio sobre este lenguaje que permite a los alumnos adquirir los conocimientos necesarios para enfrentarse a los retos que podrían proponerse en los siguientes cursos añadidos a la plataforma. 
\item Se han adquirido conocimientos avanzados de manejo de diferentes herramientas orientadas al desarrollo web y autograding de tareas, combinándose estas con otras tecnologías con el objetivo de hacer frente a los diversos retos surgidos durante la realización del proyecto. 
\item El estudiante ha podido poner en práctica técnicas y conocimientos adquiridos a lo largo de su recorrido como estudiante del Grado en Ingeniería Informática de la Universidad de Burgos, enfrentándose a problemas de desarrollo reales y aportándole experiencia aplicable en futuros proyectos a los que tendrá que hacer frente.
\end{itemize}

\section{Líneas de trabajo futuras}
Tras la finalización del proyecto, quedan pendientes las siguientes líneas de trabajo:
\begin{itemize}
\item El hecho de que los profesores puedan editar tareas desde la propia plataforma mediante la ejecución remota de Jupyter Notebooks genera el problema de que estos puedan retroceder en el árbol de archivos de Jupyter, accediendo así a las carpetas de cursos de la plataforma. Queda establecida como posible línea de trabajo la resolución de este problema, el cual, en caso de ser resuelto, posibilitaría la opción de permitir a los alumnos resolver las tareas desde la propia plataforma.
\item En el estado actual en el que se encuentra la plataforma, esta no contiene ninguna forma de comunicación directa entre profesores y alumnos. Otra posible línea de trabajo sería la adición de algún sistema de comunicación entre estos, generándose así un ambiente de trato más personal dentro de la web. 
\item Actualmente no se han incluido marcas temporales para la entrega de tareas, funcionalidad que podría resultar de utilidad e interés dentro de la plataforma; es por ello por lo que se define esta como otra futura línea de trabajo.
\item Otra posible línea de trabajo derivable de este proyecto sería la adaptación de la plataforma web a la herramienta de gestión de aprendizaje Moodle \cite{Moodle}.
\end{itemize}