\capitulo{2}{Objetivos del proyecto}

En este apartado se van a detallar los objetivos que se han perseguido durante la realización de este proyecto, dividiéndose estos en objetivos generales, objetivos técnicos y objetivos personales.

\section{Objetivos generales}

\begin{itemize}
  \item Explorar diversas herramientas orientadas a la corrección automática
  de ejercicios en Python.
  \item Seleccionar una herramienta de corrección automática de ejercicios en Python.
  \item Desarrollar una plataforma web orientada a la enseñanza la cual permita corregir de forma automática ejercicios en Python a través de la herramienta seleccionada.
  \item Diseñar un sencillo curso introductorio a Python con sus respectivas tareas autocorregibles y alojarlo en la plataforma.
\end{itemize}


\section{Objetivos técnicos}
\begin{itemize}
  \item Crear una plataforma online para la comodidad de acceso a los usuarios de esta mediante Python y Flask.
  \item Corregir de forma automática ejercicios en Python dentro de la plataforma contenidos en Notebooks de Jupyter.
  \item Permitir la creación de tareas desde la propia web mediante la ejecución de Jupyter Notebook de forma remota.
  \item Alojar el código de la plataforma dentro de un contenedor Docker para posibilitar su posterior despliegue.
  \item Adaptar la herramienta de autocorrección seleccionada (Nbgrader) para el uso específico dado en la plataforma.
   
\end{itemize}

\section{Objetivos personales}

\begin{itemize}
  \item Ofrecer una base introductoria al aprendizaje del lenguaje de programación Python.
  \item Descubrir y manejar herramientas para el desarrollo de ejercicios autocorregibles en ambientes de programación.
  \item Descubrir y manejar herramientas para la creación de entornos web.
  \item Llevar a la práctica diversos conceptos teóricos estudiados como la metodología Agile, el control de versiones durante el desarrollo de un proyecto y el trabajo en equipo.
  \item Aplicar una metodología Agile al desarrollo del proyecto.
  \item Utilizar Git como sistema de control de versiones durante el desarrollo del proyecto.
 
\end{itemize}