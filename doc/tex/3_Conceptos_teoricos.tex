\capitulo{3}{Conceptos teóricos}

En este apartada van a ser expuestos los conceptos teóricos pertenecientes a este proyecto y su desarrollo:

\section{Proceso de enseñanza - aprendizaje}
En la formación de individuos se suele hablar como un todo del proceso de enseñanza-aprendizaje\cite{EnsenanzaAprendizaje}. ¿En qué consiste este proceso? Empezaremos por definir los elementos que lo componen.

\subsection{Proceso de enseñanza}
Esta parte del proceso corresponde a las tareas de trasmisión de contenidos y conocimientos desarrolladas por el docente, para el que representa una de sus actividades más importante.

El profesor tiene en cuenta los contenidos de la materia que imparte y las estrategias didácticas para enseñar al alumno lo que ha de aprender, así como asistirle y encauzar su labor de aprendizaje. El docente modula lo que el alumno debe ir aprendiendo y cómo hacerlo, y acompaña el aprendizaje del estudiante. 


\subsection{Proceso de aprendizaje}
De acuerdo con la teoría de Piaget (1969)\cite{Piaget}\cite{Libro:Piaget}, el pensamiento es la base en la que se asienta el aprendizaje, es la manera de manifestarse la inteligencia. La inteligencia desarrolla una estructura y un funcionamiento. El propio funcionamiento modifica la estructura de la inteligencia y existe un interacción entre el individuo y el medio.

En cuanto al proceso de aprendizaje, las ideas principales de esta teoría son:

\begin{itemize}
\item El estudiante es quien lleva a cabo el aprendizaje. El profesor es un orientador o facilitador.
\item Para el aprendizaje de cualquier materia se requiere una continuidad o secuencia lógica y psicológica.
\item Han de respeterse las diferencias individuales entre los diferentes estudiantes.
\end{itemize}

Por parte del docente la enseñanza debe ser vista como el proceso de una relación personal con el estudiante en su viaje por el aprendizaje.

Es necesario comprender que el aprendizaje es personal, centrado en objetivos y, además, que precisa de constante retroalimentación. El aprendizaje debe basarse en una satisfactoria interacción entre los dos elementos principales que participan en el proceso: docente y estudiante.

Tanto el aprendizaje como la enseñanza son procesos que se presentan continuamente en la vida de todo ser humano, y no es posible hablar de ellos separadamente, pues forman una unidad conjunta.

El proceso de enseñanza-aprendizaje está compuesto por cuatro elementos: el profesor, el estudiante, la materia o contenido y los métodos didácticos. Cada uno de estos elementos influye en mayor o menor medida, dependiendo del modo en que se relacionan en un contexto determinado.

Al analizar cada uno de estos cuatro elementos, se identifican variables que influyen en el proceso enseñanza-aprendizaje:

\begin{itemize}
\item Docente: actitud, capacidad, relación con el estudiante, conocimientos de la materia, conocimientos/destrezas técnico-didácticos,  compromiso con el proceso…
\item Estudiante:  interés-motivación, capacidad, velocidad de aprendizaje, conocimientos previos, posición socioeconómica, dedicación…  
\item Materias/contendidos: complejidad, significado, importancia, relevancia práctica, interés general, proyección, …
\item Métodos didácticos: modo de presentación, técnicas-recursos-apoyos utilizados por el docente, dinámicas, actividades, sistemas de evaluación, realimentación de información sobre la evalución… 
\end{itemize}

\section{Evaluación}
El proceso de enseñanza-aprendizaje implica una variedad diferente de elementos, entre los cuales en este trabajo fin de grado nos centramos en la evaluación y la retroalimentación de información.

El tiempo de los profesores se tiene que distribuir entre varias tareas diferentes, entre las que destaca la enseñanza de conocimientos, la cual es imprescindible para dirigir a los alumnos al objetivo deseado. Cuanto mayor calidad tenga y mayor tiempo se dedique a esta actividad mejores resultados se podrán obtener en el proceso de enseñanza-aprendizaje. No obstante, los profesores pueden llegar a dedicar grandes cantidades de tiempo en actividades de corrección.

Durante las clases el profesor tiene multitud de formas de apoyar el aprendizaje de los alumnos. Una de las fundamentales es la resolución de dudas planteadas durante sus clases, lo que puede tener mucha importancia en cuestiones de difícil comprensión. 

Pero no siempre estas dudas son de igual interés para todos los alumnos y se plantea que hay alumnos o grupos que precisan un apoyo más personalizado, dedicando el profesor tiempo a tutorías y asistencia fuera del tiempo de clase. El estado actual de la tecnología ha incidido en la atención individual mediante la posibilidad de comentarios a través del uso de plataformas web y preguntas por correo electrónico. El profesorado puede dar así respuestas muy precisas y personalizadas de forma rápida.

Los profesores dedican algo de tiempo a resolver las preguntas individuales de los estudiantes y, a pesar de que esto no es muy popular, constituye una excelente manera de superar cursos difíciles. Con el advenimiento de la tecnología, la evaluación individual ha crecido significativamente. Los profesores ahorran tiempo en la recuperación de comentarios mediante el uso de plataformas web y los estudiantes se benefician enormemente de esto, ya que reciben respuestas rápidas y precisas. 

Cuando se trata de realizar la evaluación a  textos complejos de los alumnos, o temas en cuya respuesta importa el desarrollo o la interpretación como la historia o la literatura es difícil recurrir a la tecnología. Sin embargo, en no pocos casos y sobre todo en áreas técnicas, se ha conseguido una mejoría en los métodos de evaluación, reduciendo la cantidad de tiempo invertida corrigiendo. Aparte de los típicos test de evaluación a través de plataformas informáticas, destacan también los sistemas de evaluación automatizada o autograding para corrección en plataformas de enseñanza de programación.


\section{Evaluación de código informático}
En la generación de código informático para cualquier ejercicio que se plantee se suele encontrar que existe gran libertad y posibilidades de llegar al mismo resultado con diferente código. A la hora de corregir, el profesor se encuentra con dos tipos de evaluación de dicho código:

\begin{itemize}
\item En primer lugar la evaluación o corrección funcional: comprueba que el código hace realmente el trabajo solicitado, con independencia de la calidad de ese código. Es decir, si el código es eficaz para lograr el resultado.
\item En segundo lugar, existe un valoración no funcional. Aquí entran el estilo de programación, la claridad y legibilidad del código, el uso eficiente de recursos como la memoria, el tiempo de ejecución, redundancia de código, etc.
\end{itemize}

El evaluador normalmente tiene que dar un peso a cada una de estas formas de evaluación del código. No obstante, cuando se trata de programación para principiantes, las cuestiones de estilo y eficacia no suelen considerarse importantes. Es decir, corresponde aplicar sobre todo la evaluación funcional pues lo que se requiere es que el alumno llegue a conocer y manejar los elementos de lenguaje, sintaxis, estructura, funciones básicas, etc.

Para alumnos principiantes se diseñan ejercicios que requieren programas muy simples para enseñarles los conceptos básicos de la programación. Se trata de ejercicios relativamente sencillos cuya corrección es también bastante rápida. Sin embargo, el volumen de los ejercicios puede ser significativo. Si sumamos a esto que en los estadios iniciales de programación el número de alumnos inscritos suele ser elevado, el volumen de ejercicios a corregir puede llegar a ser muy alto además de tedioso. Esto ha llevado a que aparezcan enfoques nuevos para tratar de ayudar en el proceso.

\section{Autograding}
Los autograders (también denominados autoevaluadores o evaluadores automáticos) son herramientas que permiten a los profesores la evaluación automatizada de código. Su uso es bastante amplio en cursos de informática e ingeniería. También puede encontrarse en otras ramas afines como ciencia de datos y estadística e, incluso, se llega a utilizar en áreas no relacionadas con la informática.

Su uso ahorra tiempo en la calificación, en especial cuando se trata de cursos con gran número de alumnos inscritos. Ofrecen a los alumnos la evaluación de los ejercicios en tiempo real e idealmente pueden proporcionarles retroalmentación de sus resultados y/o comentarios para que puedan servirles en las siguientes etapas de su proceso de aprendizaje.

Su forma de uso común es medir la calidad del código generado por el alumno como respuesta a ejercicios propuestos mediante la comparación de los resultados de la ejecución de dicho código con el resultado esperado o la respuesta que proporciona el código de referencia que el profesor establezca como solución. Se obtendrá una calificación basada en el resultado de tal comparación.

El profesor prueba de este modo la habilidad del estudiante en los casos de programación planteados. Se puede implementar de modo que sea el estudiante el que en línea aplique el autograding y compruebe si su código es o no correcto. Asimismo otra implementación común es que el alumno envíe sus respuestas al profesor, que será quien aplique el programa de evaluación automática para obtener las calificaciones de cada alumno de grupo en los distintas tareas de ejercicios encomendadas al alumnado.

























