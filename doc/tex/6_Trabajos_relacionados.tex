\capitulo{6}{Trabajos relacionados}

Como se vio en los aspectos relavantes del desarrollo del proyecto, al realizar una investigación inicial acerca de las herramientas de autograding se descubrieron diversas plataformas orientadas a la corrección automática de ejercicios en lenguajes de programación como CodeGrade o CodingRooms. A continuación veremos otras plataformas y proyectos que siguen un formato similar:

\section{Plataformas}

\subsubsection{Codequiry}
Codequiry\cite{tool:Codequiry}\cite{tool:CodequiryAutograding} es una plataforma orientada a la detección de plagio de código la cual incluye, entre otras herramientas, la calificación de tareas de programación de forma automatizada, soportando hasta 20 lenguajes de programación distintos.
\begin{itemize}
\item \href{https://codequiry.com/}{Página oficial de Codequiry}
\item \href{https://codequiry.com/auto-grading-programming}{Sección orientada al autograding} 
\end{itemize}

\subsubsection{Codio}
Codio\cite{tool:Codio}\cite{tool:CodioAutograding} es una plataforma en la nube dedicada a la creación, calificación, testeo y calificación tanto manual como automática de tareas de programación.
La misión de Codio es la de impartir contenido educacional de alta calidad sobre la ciencia de datos, accesible por estudiantes de todo el mundo.
\begin{itemize}
\item \href{https://www.codio.com/}{Página oficial de Codio}
\item \href{https://www.codio.com/features/auto-grading}{Sección orientada al autograding}
\end{itemize}

\section{Proyectos}

\subsubsection{Desarrollo de un Sistema de Corrección Automática de Programas Haskell}
Se trata de un Trabajo de Fin de Grado realizado por Meritxell Sáez Povedano\cite{MeritxellSaezPovedano} en 2016 para la Universidad Politécnica de Valencia, orientado a la corrección automática de programas escritos en el lenguaje de programación Haskell.

A diferencia de nuestro proyecto el cual ha seguido una orientación didáctica basada en cursos, adaptando herramientas de autograding a dicho propósito, este proyecto optaba por la extensión del ya existente sistema de evaluación semi-automática de ejercicios en Java ''ASys''\cite{tool:ASys} con el fin de posibilitar la evaluación de ejercicios en Haskell.

\begin{itemize}
\item \href{http://personales.upv.es/josilga/ASys/about.html}{El sistema ASys}
\item \href{https://riunet.upv.es/handle/10251/72442?show=full}{Proyecto de Meritxell Sáez Povedano} 
\end{itemize}



\subsubsection{Aplicación web para el análisis automático de prácticas realizadas en el lenguaje Java}
Este es el Trabajo de Fin de Grado de Álvaro Vázquez\cite{AlvaroVazquez} para la Universidad de Burgos en 2019. En él se desarrolla una aplicación web de corrección automática de prácticas en el lenguaje Java ligada a la herramienta de gestión de aprendizaje Moodle, sistema muy similar al desarrollado en nuestro proyecto basado en un lenguaje de programación diferente.